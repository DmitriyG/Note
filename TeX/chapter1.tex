   %Глава 1. Теоретическое обоснование дипломного проекта.
    %Анализ. Классификация. Сравнение.
    %Анализ предметной области. Т.е. компьютерной сети.
    %Рассмотрение различных способов моделирования. Обоснование выбора используемого.

    %План.
      %Введение.
      %Моделирование
\chapter{}
\section{Классификация видов моделирования.}

\subsection{Классификационные признаки.}

 В качестве одного из первых признаков классификации видов моделирования можно выбрать степень полноты модели и разделить модели в соответствии с этим на полные, неполные и приближенные. В основе полного моделирования лежит полное подобие, которо проявляется как во времени, так и в пространстве. Для неполного моделирования характерно неполное подобие модели изучаемому объекту. В основе приближенного моделирования лежит приближенное подобие, при котором некоторые стороны функционирования реального объекта не моделируются совсем. Классификация видов моделирования приведена на рисунке 1.

    В зависимости от характера изучаемых процессов в системе S все виды моделирования могут быть разделены на детерминированные и стохастические, статические и динамические, дискретные, непрерывные и дискретно-непрерывные. Детерминированное моделирование отображает детерминированные процессы, т.е. процессы, в которых предполагается отсутствие всяких случайных воздействий; стохастическое моделирования отображает вероятностные процессы и события. В этом случае анализируется ряд реализаций случайного процесса и оцениваются средние характеристики, т.е. набор однородных реализаций. Статическое моделирования служит для описания поведения объекта в какой-либо момент времени, ф динамическое моделирование отражает поведение объекта во времени. Дискретное моделирование служит для описания процессов, которые предполагаются дискретными, соответственно непрерывное моделирование используется для случаев, когда хотят выделить наличие как дискретных, так и непрерывных процессов.

    В зависимости от формы представления объекта (системы S) можно выделить мысленное и реальное моделирование.

    Мысленное моделирование часто является единственным способом моделирования объектов, которые либо практически нереализуемы а заданном интервале времени, либо существуют вне условий, возможных для их физического создания. Мысленное моделирование может быть реализовано в виде наглядного, символического и математического.

    При наглядном моделировании на базе представлений человека о реальных объектах создаются различные наглядные модели, отображающие явления и процессы, протекающие в объекте. В основу гипотетического моделирования исследователем закладывается некоторая гипотеза о закономерностях протекания процесса в реальном объекте, которая отражает уровень знаний исследователя об объекте и базируется на причинно-следственных связях между входом и выходом изучаемого объекта. Гипотетическое моделирование используется, когда знаний об объекте недостаточно для построения формальных моделей.

    Аналоговое моделирование основывается на применении аналогий различных уровней. Наивысшим уровнем является полная аналогия, имеющая место только для достаточно простых объектов. С усложнением объекта используют аналогии последующих уровней, когда аналоговая модель обображает несколько либо только одну сторону функционирования объекта.

    Существенное место при мысленном наглядном моделировании занимает макетирование. Мысленный макет может применяться в случаях, когда протекающие в реальном объекте процессы не поддаются физическому моделированию, либо может предшествовать проведению других видов моделирования. В основе построения мысленных макетов также лежат аналогии, однако обычно базирующиеся на причинно-следственных связях между явлениями и процессами в объекте. Если ввести условное обозначение отдельных понятий, т.е. знаки, а также определенные операции между этими знаками, то можно реализовать знаковое моделирование и с помощью знаков отображать набор понятий -- составлять отдельные цепочки из слов и предложений.

    В основе языкового моделирования лежит некоторый тезаурус. Последний образуется из набора входящих понятий, причем этот набор должен быть фиксированным. Следует отметить, что между тезаурусом и обычным словарем имеются принципиальные различия. Тезаурус - словарь, который очищен от неоднозначности, т.е. в нем каждому слову может соответствовать лишь единственное понятие, хотя в обычном словаре одному слову могут соответствовать несколько понятий.

    Символическое моделирование представляет собой искусственный процесс создания логического объекта, который замещает реальный и выражает основные свойства его отношений с помощью определенной системы знаков или символов.

\subsection{Математическое моделирование.}

 Для исследования характеристик процесса функционирования любой системы S математическими методами, включая и машинные, должна быть проведена формализация этого процесса, т.е. построена математическая модель.

    Под математическим моделированием понимают процесс установления соответствия данному реальному объекту некоторого математического объекта, называемого математической моделью, и исследование этой модели, позволяющее получать характеристики рассматриваемого реального объекта. Вид математического модели зависит как от природы реального объекта, так и задач исследования объекта и требуемой достоверности и точности решения этой задачи. Любая математическая модель, как и всякая другая описывает реальный объект лишь с некоторой степенью прближения к действительности. Математическое моделирование для исследвания характеристик процесса функционирования систем можно разделить на аналитическое, имитационное и комбинированное.

    Для аналитического моделирования характерно то, что процессы функционирования элементов системы записываются в виде некоторых функциональных соотношений (алгебраических, интегродифференциальных, конечно-разностных и.т.п.) или логических условий. Аналитическая модель может быть исследована следующими методами:
\begin{enumerate}
  \item аналитическим, когда стремятся получить в общем виде явные зависимости для искомых характеристик;
  \item численным, когда, не умея решать уравнений в общем виде, стремятся получить числовые результаты при конкретных начальных данных;
  \item качественным, когда, не имея решения в явном виде, можно найти некоторые свойства решения.
\end{enumerate}

    Наиболее полное исследование процесса функционирования системы можно провести, если известны явные зависимости, связывающие искомые характеристики с начальными условиями, параметрами и переменными системы S. Однако такие зависимости удается получить только лоя сравнительно простых систем. При усложнении систем исследование их аналитическим методом наталкивается на значительные трудности, которые часто бывают непреодолимыми. Поэтому, желая использовать аналитический метод, в этом случае идут на существенное упрощение первоначальной модели, чтобы иметь возможность изучить хотя бы общие свойства системы. Такое исследование на упрощенной модели аналитическим методом помогает получить ориентировочные результаты для определения более точных оценок другими методами. Численный метод позволяет исследовать по сравнению с аналитическим методом более широкий класс систем, но при этом полученные решения носят частный характер. Численный метод особенно эффективен при использовании ЭВМ.

    В отдельных случаях исследования системы могут удовлетворять и те выводы, которые можно сделать про использовании качественного метода анализа математической модели. Такие качественные методы широко используются, например, в теории автоматического управления для оценки эффективности различных вариантов систем управления.

    В настоящее время распространены методы машинной реализации исследования характеристик процесс функционирования больших систем. Для реализации математической модели на ЭВМ необходимо построить соответствующий моделирующий алгоритм.

    При имитационном моделировании реализующий модель алгоритм воспроизводит процесс функционирования системы S во времени, причем имитируются элементарные явления, составляющие процесс, с сохранением их логической структуры и последовательности протекания во времени , что позволяет по исходным данным получить сведения о состояниях процесса в определенные моменты времени, дающие возможность оценить характеристики системы S.

    Основным преимуществом имитационного моделирования по сравнению с аналитическим является возможность решения более сложных задач. Имитационные модели позволяют достаточно просто учитывать такие факторы, как наличие дискретных и непрерывных элементов, нелинейные характеристики элементов системы, многочисленные случайные воздействия и др., котоые часто создают трудности при аналитических исследованиях. В настоящее время имитационное моделирование -- наиболее эффективный метод исследования больших систем, а часто и единственный практически доступный метод получения информации о поведении системы, особенно на этапе ее проектирования.
    Когда результаты, полученные про воспроизведении на имитационной модели процесса функционирования системы S, являются реализациями случайных величин и функций, тогда для нахождения характеристик процесса требуется его многократное воспроизведение с последующей статистической обработкой информации и целесообразно в качестве метода машинной реализации имитационной модели использовать метод статистического моделирования. Первоначально был разработан метод статистических испытаний, представляющий собой численный метод, который применялся для моделирования случайных величин и функций, вероятностные характеристики которых совпадали с решениями аналитических задач (такая процедура получила название метода Монте-Карло). Затем этот прием стали применять и для машинной имитации с целью исследования характеристик процессов функционирования систем, подверженных случайным воздействиям, т.е. появился метод статистического моделирования. Таким образом, методом статистического моделирования называют метод машинной реализации имитационной модели, а методом статистических испытаний -- численный метод решения аналитической задачи.

    Метод имитационного моделирования позволяет решать задачи анализа больших систем, включая задачи оценки: вариантов структуры систем, эффективности различных алгоритмов управления системой, влияния изменения различных параметров системы. Имитационное моделирование может быть положено также в основу структурного. алгоритмического и параметрического синтеза больших систем, когда требуется создать систему, с заданными характеристиками при определенных ограничениях, которая является оптимальной по некоторым критериям оценки эффективности.

    При решении задач машинного синтеза систем на основе их имитационных моделей помимо разработки моделирующих алгоритмов для анализа фиксированной системы необходимо так же разработать алгоритмы поиска оптимального варианта системы. Далее в методологии машинного моделирования будем различать два основных раздела: статику и динамику, -- основным содержанием которых являются соответственно вопросы анализа и синтеза систем, заданных моделирующими алгоритмами.

    Комбинированное (аналитико-имитационное) моделирование при анализе и синтезе систем позволяет объединить достоинства аналитического и имитационного моделирования. При построении комбинированных моделей проводится предварительная декомпозиция процесса функционирования объекта на составляющие подпроцессы и для тех из них, где это возможно, используются аналитические модели, а для остальных подпроцессов строятся имитационные модели. Такой комбинированный подход позволяет охватить качественно новые классы систем, которые не могут быть исследованы с использованием только аналитического и имитационного моделирования в отдельности.

    Другие виды моделирования.

    При реальном моделировании используется возможность исследования различных характеристик либо на реальном объекте целиком, либо на его части. Такие исследования могут проводиться как на объектах, работающих в нормальных режимах, так и при организации специальных режимов, для оценки интересующих исследователя характеристик. Реальное моделирование является наиболее адекватным, но при этом его возможности с учетом особенностей реальных объектов ограничены.

    Натурным моделированием называют проведение исследования на реальном объекте с последующей обработкой результатов эксперимента на основе теории подобия. При функционировании объекта в соответствии с поставленной целью удается выявить закономерности протекания реального процесса. Надо отметить, что такие разновидности натурного эксперимента, как производственный эксперимент и комплексные испытания, обладают высокой степенью достоверности.

    Другим видом реального моделирования является физическое, отличающееся от натурного тем, что исследование проводится на установках, которые сохраняют природу явлений и обладают физическим подобием. А процессе физического моделирования задаются некоторые характеристики внешней среды и исследуется поведение либо реального объекта, либо его модели при заданных или создаваемых искусственно воздействиях внешней среды. Физическое моделирование может протекать в реальном и нереальном (псевдореальном) масштабах времени, а так же рассматриваться без учета времени. В последнем случае изучению подлежат так называемые "Замороженные" процессы, которые фиксируются в некоторый момент времени.

\subsection{Математическое моделирование.}

\subsubsection{Формальная модель объекта.}

Модель объекта моделирования, т. е. системы S, можно представить в виде множества величин, описывающих процесс функционирования реальной системы и образующих в общем случае следующие подмножества:

\begin{enumerate}
  \item совокупность входных воздействий на систему
      \begin{center}
        $x_{i} \in X, i = \sup{1, n_{X}}$; х,еХ, i=l , пх;
      \end{center}

  \item совокупность воздействий внешней среды
     \begin{center}
       $\upsilon_{l} \in V l = \sup{1, n_{V}}$;
     \end{center}

  \item совокупность внутренних (собственных) параметров системы
     \begin{center}
       $h_{k} \in H, k = \sup{1, n_{H}}$;
     \end{center}

  \item совокупность выходных характеристик системы
    \begin{center}
      $y_{j} \in Y, j = \sup{1, n_{Y}}$
    \end{center}

\end{enumerate}

  При этом в перечисленных подмножествах можно выделить управляемые и неуправляемые переменные. В общем случае $x_{i}$, $\upsilon_{l}$, $h_{k}$, $y_{j}$ являются элементами непересекающихся подмножеств и содержат как детерминированные, так и стохастические составляющие. При моделировании системы S входные воздействия, воздействия внешней среды Е и внутренние параметры системы являются независимыми (экзогенными) переменными, которые в векторной форме имеют соответственно вид $\vec{x}(t) = (x_{1}(t), x_{2}(t), ..., x_{nX}(t))$; $\vec{v}(t) = (v_{1}(t), v_{2}(t), ... , v_{nV}(t))$; $\vec{h}(t) = (h_{1}(t), h_{2}(t), ... , h_{nH}(t))$, а выходные характеристики системы являются зависимыми (эндогенными) переменными и в векторной форме имеют вид $\vec{y}(t) = (y_{1}(t), y_{2}(t), ... , y_{nY}(t))$.

  Процесс функционирования системы S описывается во времени оператором Fs, который в общем случае преобразует экзогенные переменные в эндогенные в соответствии с соотношениями вида

  \begin{center}
    $\vec{y}(t) = F_{s}(\vec{x}, \vec{\upsilon}, \vec{h}, t)$
  \end{center}

  Совокупность зависимостей выходных характеристик системы от времени $y_{j}(t)$ для всех видов $j = \sup{1, n_{y}}$ называется выходной траекторией $\vec{y}(t)$. Зависимость (2.1) называется законом функционирования системы S и обозначается $F_{s}$. В общем случае закон функционирования системы $F_{s}$ может быт задан в виде функции, функционала, логических условий, в алгоритмической и табличной формах или в виде словесного правила соответствия. Весьма важным для описания и исследования системы S является понятие алгоритма функционирования $A_{s}$, под которым понимается метод получения выходных характеристик с учетом входных воздействий $\vec{x}(t)$, воздействий внешней среды $\vec{\upsilon}(t)$ и собственных параметров системы $\vec{h}(t)$. Очевидно, что один и тот же закон функционирования $F_{s}$ системы S может быть реализован различными способами, т. е. с помощью множества различных алгоритмов функционирования $A_{s}$. Соотношения (2.1) являются математическим описанием поведения объекта (системы) моделирования во времени t, т. е. отражают его динамические свойства. Поэтому математические модели такого вида принято называть динамическими моделями (системами).

  Для статических моделей математическая модель (2.1) представляет собой отображение между двумя подмножествами свойств моделируемого объекта Y и \{X, V, Н\}, что в векторной форме может быть записано как

  \begin{center}
      $\vec{y} = f(\vec{x}, \vec{\upsilon}, \vec{h})$.(2.2)
  \end{center}

  Соотношения (2.1) и (2.2) могут быть заданы различными способами: аналитически (с помощью формул), графически, таблично и т. д. Такие соотношения в ряде случаев могут быть получены через свойства системы S в конкретные моменты времени, называемые состояниями. Состояние системы S характеризуется векторами

  \begin{center}
    $\vec{z}' = (z_{1}', z_{2}', ... , z_{k}')$ и $\vec{z}'' = (z_{1}'', z_{2}'', ... , z_{k}'')$,
  \end{center}


  где $z_{1}'= z_{1}(t'), z_{2}' = z_{2}(t'), ... , z_{k}' = z_{k}(t')$ в момент $t' \in (t_{0}, T)$, $z_{1}'' = z_{1}(t''), z_{2}'' = z_{2}(t''), ... , z_{k}'' = z_{k}(t'')$  в момент $t'' \in (t_{0}, T)$ и т. д., $k = \sup{1, n_{z}}$. Если рассматривать процесс функционирования системы S как последовательную смену состояний $z_{1}(t), z_{2}(t), ... , z_{k}(t)$, то они могут быть интерпретированы как координаты точки в k-мерном фазовом пространстве, причем каждой реализации процесса будет соответствовать некоторая фазовая траектория. Совокупность всех возможных значений состояний $\{\vec{z}\}$ называется пространством состояний объекта моделирования Z, причем $z_{k} \in Z$.

  Состояния системы S в момент времени $t_{0} < t* <= T $ полностью определяются начальными условиями $\vec{z}^{0} = (z^{0}_{1}, z^{0}_{2}, ... , z^{0}_{k})$ [где $z^{0}_{1} = z_{1}(t_{0}), z^{0}_{2} = z_{2}(t_{0}), ... , z^{0}_{k} = z_{k}(t_{0})$], входными воздействиями $\vec{x}(t)$, внутренними параметрами $\vec{h}(t)$ и воздействиями внешней среды $\vec{\upsilon}(t)$, которые имели место за промежуток времени t* -- $t_{0}$, с помощью двух векторных уравнений

  \begin{center}
     $z(t) = \Phi(\vec{z}_{0}, \vec{x}, \vec{\upsilon}, \vec{h}, t)$ (2.3)

     $\vec{y}(t) = F(\vec{z}, t)$ (2.4)
  \end{center}


  Первое уравнение по начальному состоянию $\vec{z}^{0}$ и экзогенным переменным $\vec{x}, \vec{\upsilon}, \vec{h}$ определяет вектор-функцию $\vec{z}(t)$, а второе по полученному значению состояний $\vec{z}(t)$ — эндогенные переменные на выходе системы $\vec{y}(t)$. Таким образом, цепочка уравнений объекта «вход — состояния — выход» позволяет определить характеристики системы

  \begin{center}
    $\vec{y}(t) = F[\Phi(\vec{z}^{0}, \vec{x}, \vec{\upsilon}, \vec{h}, t)]$. (2.5)
  \end{center}

  В общем случае время в модели системы S может рассматриваться на интервале моделирования (0, Т) как непрерывное, так и дискретное, т. е. квантованное на отрезки длиной $\delta t$ временных единиц каждый, когда $T = m\delta t$, где $m = \sup{1,m_{T}}$ — число интервалов дискретизации.

  Таким образом, под математической моделью объекта (реальной системы) понимают конечное подмножество переменных \{ $\vec{x}(t), \vec{\upsilon}(t), \vec{h}(t) $\}вместе с математическими связями между ними и характеристиками $\vec{y}$.

  Если математическое описание объекта моделирования не содержит элементов случайности или они не учитываются, т. е. если можно считать, что в этом случае стохастические воздействия внешней среды $\vec{\upsilon}(t)$ и стохастические внутренние параметры $\vec{h}(t)$ отсутствуют, то модель называется детерминированной в том смысле, что характеристики однозначно определяются детерминированными входными воздействиями

  \begin{center}
    $\vec{y}(t) = f(\vec{x}, t)$. (2.6)
  \end{center}

  Очевидно, что детерминированная модель является частным случаем стохастической модели.

\subsubsection{Математическая схема.}

  Математическую схему можно определить как звено при переходе от содержательного к формальному описанию процесса функционирования системы с учетом воздействия внешней среды, т. е имеет место цепочка «описательная модель — математическая схема — математическая [аналитическая или (и) имитационная] модель».

  Каждая конкретная система S характеризуется набором свойств под которыми понимаются величины, отражающие поведение моделируемого объекта (реальной системы) и учитывающие условия ее функционирования во взаимодействии с внешней средой (системой) Е. При построении математической модели системы необходимо решить вопрос об ее полноте. Полнота модели регулируется в основном выбором границы «система S — среда Е». Также должна быть решена задача упрощения модели, которая помогает выделить основные свойства системы, отбросив второстепенные. Причем отнесение свойств системы к основным или второстепенным существенно зависит от цели моделирования системы (например, анализ вероятностно-временных характеристик процесса функционирования системы, синтез структуры системы и т. д.).

  Приведенные математические соотношения представляют собой математические схемы общего вида и позволяют описать широкий класс систем. Однако в практике моделирования объектов в области системотехники и системного анализа на первоначальных этапах исследования системы рациональнее использовать типовые математические схемы: дифференциальные уравнения, конечные и вероятностные автоматы, системы массового обслуживания, сети Петри и т. д.

  Не обладая такой степенью общности, как рассмотренные модели, типовые математические схемы имеют преимущества простоты и наглядности, но при существенном сужении возможностей применения. В качестве детерминированных моделей, когда при исследовании случайные факторы не учитываются, для представления систем, функционирующих в непрерывном времени, используются дифференциальные, интегральные, интегродифференциальные и другие уравнения, а для представления систем, функционирующих в дискретном времени,— конечные автоматы и конечно-разностные схемы. В качестве стохастических моделей (при учете случайных факторов) для представления систем с дискретным временем используются вероятностные автоматы, а для представления системы с непрерывным временем — системы массового обслуживания и т. д.

  Перечисленные типовые математические схемы, естественно, не могут претендовать на возможность описания на их базе всех процессов, происходящих в больших информационно-управляющих системах. Для таких систем в ряде случаев более перспективным является применение агрегативных моделей. Агрегативные модели (системы) позволяют описать широкий круг объектов исследования с отображением системного характера этих объектов. Именно при агрегативном описании сложный объект (система) расчленяется на конечное число частей (подсистем), сохраняя при этом связи, обеспечивающие взаимодействие частей.

  Таким образом, при построении математических моделей процессов функционирования систем можно выделить следующие основные подходы: непрерывно-детерминированный (например, дифференциальные уравнения); дискретно-детерминированный (конечные автоматы); дискретно-стохастический (вероятностные автоматы); непрерывно-стохастический (системы массового обслуживания); обобщенный, или универсальный (агрегативные системы).

 \subsubsection{D -- схемы}

 Обычно в таких математических моделях в качестве независимой переменной, от которой зависят неизвестные искомые функции, служит время t. Тогда математическое соотношение для детерминированных систем (2.6) в общем виде будет

 \begin{center}
    $\vec{y}'= \vec{f}(\vec{y}, t), \vec{y}(t_{0}) = \vec{y}_{0}$
 \end{center}

 где $\vec{y}' = \dfrac{d\vec{y}}{dt}$,$\vec{y} = (y_{1}, y_{2}, ... , y_{n})$ и $\vec{f} = (f_{1}, f_{2}, ... , f_{n})$ -- n-мерные векторы;$f(\vec{y}, \vec{t})$ -- вектор-функция, которая определена на некотором (n + 1)-мерном $(\vec{y}, t)$ множестве и является непрерывной. Так как математические схемы такого вида отражают динамику изучаемой системы, т. е. ее поведение во времени, то они называются D-схемами (англ. dynamic).

 В простейшем случае обыкновенное дифференциальное уравнение имеет вид

  \begin{center}
    $y' = f(y, t)$
  \end{center}
 Наиболее важно для системотехники приложение D-схем в качестве математического аппарата в теории автоматического управления.

 \subsubsection{Дискретно-детерминированные модели. (F -- схемы)}

 Особенности дискретно-детерминированного подхода на этапе формализации процесса функционирования систем рассмотрим на примере использования в качестве математического аппарата теории автоматов. Теория автоматов — это раздел теоретической кибернетики, в котором изучаются математические модели — автоматы. На основе этой теории система представляется в виде автомата, перерабатывающего дискретную информацию и меняющего свои внутренние состояния лишь в допустимые моменты времени. Понятие «автомат» варьируется в зависимости от характера конкретно изучаемых систем, от принятого уровня абстракции й целесообразной степени общности.

 Автомат можно представить как некоторое устройство (черный ящик), на которое подаются входные сигналы и снимаются выходные и которое может иметь некоторые внутренние состояния. Конечным автоматом называется автомат, у которого множество внутренних состояний и входных сигналов (а следовательно, и множество выходных сигналов) являются конечными множествами. Абстрактно конечный автомат (англ. finite automata) можно представить как математическую схему (F-схему), характеризующуюся шестью элементами: конечным множеством X входных сигналов (входным алфавитом); конечным множеством Y выходных сигналов (выходным алфавитом); конечным множеством Z внутренних состояний (внутренним алфавитом или алфавитом состояний); начальным состоянием $z_{0}, z_{0} \in Z $; функцией переходов $\phi(z, x)$; функцией выходов $\psi(z, x)$. Автомат, задаваемый F-схемой: $F = <Z, X, Y, \phi, \psi, z_{0}>$, -- функционирует в дискретном автоматном времени, моментами которого являются такты, т. е. примыкающие друг к другу равные интервалы времени, каждому из которых соответствуют постоянные значения входного и выходного сигналов и внутренние состояния.

 Абстрактный конечный автомат имеет один входной и один выходной каналы. В каждый момент t=0, 1, 2, ... дискретного времени F-автомат находится в определенном состоянии z(t) из множества Z состояний автомата, причем в начальный момент времени t = 0 он всегда находится в начальном состоянии $z(0) = z_{0}$ . В момент t, будучи в состоянии z(j), автомат способен воспринять на входном канале сигнал $x(t) \in X$ и выдать на выходном канале сигнал $y{t} = \psi[z(t), x{t}]$, переходя в состояние $z(t+1) = \phi[z(t), x(t)], z(t) \in Z, y(t) \in Y$. Абстрактный конечный автомат реализует некоторое отображение множества слов входного алфавита X на множество слов выходного алфавита Y.

 \subsubsection{Дискретно-стохастические модели. (P - схемы)}

  В общем виде вероятностный автомат (англ. probabilistic automat) можно определить как дискретный потактный преобразователь информации с памятью, функционирование которого в каждом такте зависит только от состояния памяти в нем и может быть описано статистически.Применение схем вероятностных автоматов (Р-схем) имеет важное значение для разработки методов проектирования дискретных систем, проявляющих статистически закономерное случайное поведение, для выяснения алгоритмических возможностей таких систем и обоснования границ целесообразности их использования, а также для решения задач синтеза по выбранному критерию дискретных стохастических систем, удовлетворящих заданным ограничениям.

  Введем математическое понятие Р-автомата, используя понятия, введенные для F-автомата. Рассмотрим множество G, элементами которого являются всевозможные пары $x_{i},  z{k}$, где $x_{i}$ и $z_{k}$, элементы входного подмножества X и подмножества состояний Z соответственно. Если существуют две такие функции $\phi$ и $\psi$, то с их помощью осуществляются отображения $G \to Z$ и $G \to Y$, то говорят, что $F = <Z, X, Y, \phi, \psi>$определяет автомат детерминированного типа. Введем в рассмотрение более общую математическую схему.Пусть $\Phi$ — множество всевозможных пар вида $(z_{k}, y_{j})$, где $y_{j}$ элемент выходного подмножества Y. Потребуем, чтобы любой элемент множества G индуцировал на множестве Ф некоторый закон распределения следующего вида:


  \begin{tabular}{ccccccc}
    Элементы из $\Phi$ &  ... & $(z_{1}, y{1})$... & $(z_{2}, y_{2})$... & ... & $(z_{K}, y_{J-1}$ & $(z_{k}, y_{j})$ \\
    $(x_{j}, z_{k})$ & ... & $b_{1 1}$ & $b_{1 2}$ & ... & $b_{K (J-1)}$ & $b_{K J}$ \\
  \end{tabular}

  При этом $\sum_{k=1}^{K}\sum_{j=1}^{J} b_{kj} = 1$ где  $b_{kj}$ — вероятности перехода автомата в состояние $z_{k}$ и появления на выходе сигнала $y_{j}$ если он был в состоянии $z_{k}$ и на его вход в этот момент времени поступил сигнал $x_{i}$. Число таких распределений, представленных в виде таблиц, равно числу элементов множества G. Обозначим множество этих таблиц через В. Тогда четверка элементов $P = <Z, X, Y, B>$ называется вероятностным автоматом (Р-автоматом)

\subsubsection{Непрерывно-стохастические модели.(Q-схемы)}

  Непрерывно-стохастический подход находит свое применение в качестве математических схем систем массового обслуживания. Системы массового обслуживания представляют собой класс математических схем, разработанных в теории массового обслуживания и различных приложениях для формализации процессов функционирования систем, которые по своей сути являются процессами обслуживания.

  В качестве процесса обслуживания могут быть представлены различные по своей физической природе процессы функционирования экономических, производственных, технических и других систем, например потоки поставок продукции некоторому предприятию, потоки деталей и комплектующих изделий на сборочном конвейере цеха, заявки на обработку информации ЭВМ от удаленных терминалов и т. д. При этом характерным для работы таких объектов является случайное появление заявок (требований) на обслуживание и завершение обслуживания в случайные моменты времени, т. е. стохастический характер процесса их функционирования. Остановимся на основных понятиях массового обслуживания, необходимых для использования Q-схем, как при аналитическом, так и при имитационном.

  В любом элементарном акте обслуживания можно выделить две основные составляющие: ожидание обслуживания заявкой и собственно обслуживание заявки. Это можно изобразить в виде некоторого i-го прибора обслуживания $\text{П}_{i}$, состоящего из накопителя заявок $H_{i}$, в котором может одновременно находиться $l_{i} = \sup{0, L^{H}_{i}}$ заявок, $L^{i}_{H}$  — емкость i-го накопителя, и канала обслуживания заявок (или просто канала) $K_{i}$. На каждый элемент прибора обслуживания $П_{i}$ поступают потоки событий: в накопитель $H_{i}$ — поток заявок $w_{i}$, на канал $K_{i}$ — поток обслуживании $u_{i}$.

  Потоком событий называется последовательность событий, происходящих одно за другим в какие-то случайные моменты времени. Различают потоки однородных и неоднородных событий. Поток событий называется однородным, если он характеризуется только моментами поступления этих событий (вызывающими моментами) и задается последовательностью $\{t_{n}\} = \{0 <= t_{1} <= t_{2} <= ... <= t_{n} <= ...\}$, где $t_{n}$ -- момент наступления n-го события -- неотрицательное вещественное число. Однородный поток событий также может быть задан в виде последовательности промежутков времени между n-м и (n—1)-м событиями $\{\tau_{n}\}$, которая однозначно связана с последовательностью вызывающих моментов $\{t_{n}\}$ , где $\tau_{n} = t_{n} - t_{n} - t_{n-1}$, $t_{0}$, т.е. $\tau_{1} = t_{1}$.

  Потоком неоднородных событий называется последовательность $(t_{n}, f_{n})$, где $t_{n}$ -- вызывающие моменты; $f_{n}$ -- набор признаков события. Например, применительно к процессу обслуживания для неоднородного потока заявок заданы принадлежность к тому или иному источнику заявок, наличие приоритета, возможность обслуживания тем или иным типом канала и.т.п.

\subsubsection{Сетевые модели. (N - схемы.)}

  В практике моделирования объектов часто приходится решать задачи, связанные с формализованным описанием и анализом причинно-следственных связей в сложных системах, где одновременно параллельно протекает несколько процессов. Самым распространенным в настоящее время формализмом, описывающим структуру и взаимодействие параллельных систем и процессов, являются сети Петри.

  Теория сетей Петри развивается в нескольких направлениях: разработка математических основ, структурная теория сетей, различные приложения (параллельное программирование, дискретные динамические системы и т. д.).

  Формально сеть Петри (N-схема) задается четверкой вида

  \begin{center}
    $N = <B, D, I, O>$
  \end{center}

  где В — конечное множество символов, называемых позициями, $B \ne \oslash$; D — конечное множество символов, называемых переходами, $D \ne \oslash$ , $B \cap D \ne \oslash$; I—входная функция (прямая функция инцидентности), $I \colon B \times D\to \{0, 1\}$; О — выходная функция (обратная функция инцидентности), $O \colon D \times B \to \{0, 1\} $. Таким образом, входная функция I отображает переход $d_{I}$ в множество входных позиций $ b_{i} \in I(d_{j})$, а выходная функция О отображает переход $d_{j}$ в множество выходных позиций $b_{i} \in D(d_{j})$. Для каждого перехода $d_{j} \in D$ можно определить множество входных позиций перехода $I(d_{j})$ и выходных позиций перехода $O(d_{j})$

%  \begin{left}
%    $I(d_{j}) = \{b_{i} \in B|I(b_{i}, d_{j}) = 1\}$
%  \end{left}

%  \begin{right}
%    $i = \sup{1,n}, j = \sup{1, m}, n = |B|, m = |D|$
%  \end{right}

%  \begin{left}
%    $O(d_{j}) = \{b_{i} \in B|I(d_{j}, b_{i}) = 1\}$
%  \end{left}

  Аналогично, для каждого перехода $b_{i} \in B$ вводятся определения множества входных переходов позиции $I(b_{i})$ и множества выходных переходов позиции $O(b_{i})$:

  \begin{center}
    $I(b_{i}) = \{d_{j} \in D|I(d_{j}, b_{i}) = 1 \}$
    `
    $O(b_{i}) = \{d_{j} \in D|O(b_{i}, d_{j}) = 1 \}$
  \end{center}

  Графически N-схема изображается в виде двудольного ориентированного мультиграфа, представляющего собой совокупность позиций и переходов. Как видно из этого рисунка, граф N-схемы имеет два типа узлов: позиции и переходы, изображаемые 0 и 1 соответственно. Ориентировочные дуги соединяют позиции и переходы, причем каждая дуга направлена от элемента одного множества (позиции или перехода) к элементу другого множества (переходу или позиции). Граф N-схемы является мультиграфом, так как он допускает существование кратных дуг от одной вершины к другой.

\subsubsection{Агрегатное моделирование. (А-схемы)}

  Анализ существующих средств моделирования систем и задач, решаемых с помощью метода моделирования на ЭВМ, неизбежно приводит к выводу, что комплексное решение проблем, возникающих в процессе создания и машинной реализации модели, возможно лишь в случае, если моделирующие системы имеют в своей основе единую формальную математическую схему, т. е. А-схему. Такая схема должна одновременно выполнять несколько функций: являться адекватным математическим описанием объекта моделирования, т. е. системы S, служить основой для построения алгоритмов и программ при машинной реализации модели М, позволять в упрощенном варианте (для частных случаев) проводить аналитические исследования.

  Приведенные требования в определенной степени противоречивы. Тем не менее в рамках обобщенного подхода на основе А-схем удается найти между ними некоторый компромисс.

  По традиции, установившейся в математике вообще и в прикладной математике в частности, при агрегативном подходе сначала дается формальное определение объекта моделирования -- агрегативной системы, которая является математической схемой, отображающей системный характер изучаемых объектов. При агрегативном описании сложный объект (система) разбивается на конечное число частей (подсистем), сохраняя при этом связи, обеспечивающие их взаимодействие. Если некоторые из полученных подсистем оказываются в свою очередь еще достаточно сложными, то процесс их разбиения продолжается до тех пор, пока не образуются подсистемы, которые в условиях рассматриваемой задачи моделирования могут считаться удобными для математического описания. В результате такой декомпозиции сложная система представляется в виде многоуровневой конструкции из взаимосвязанных элементов, объединенных в подсистемы различных уровней.

  В качестве элемента А-схемы выступает агрегат, а связь между агрегатами (внутри системы S и с внешней средой Е) осуществляется с помощью оператора сопряжения R. Очевидно, что агрегат сам может рассматриваться как А-схема, т. е. может разбиваться на элементы (агрегаты) следующего уровня.

  Любой агрегат характеризуется следующими множествами: моментов времени Т, входных X и выходных Y сигналов, состояний Z в каждый момент времени T. Состояние агрегата в момент времени $t \in T$ обозначается как $z(t) \in Z$, а входные и выходныесигналы — как $x(t) \in X$ и $y(t) \in Y$ соответственно.

  Будем полагать, что переход агрегата из состояния $z(t_{1})$ в состояние $z(t_{2}) \ne z(t_{1})$ происходит за малый интервал времени, т. е. имеет место скачок $\delta z$. Переходы агрегата из состояния $z(t_{1})$ в $z(t_{2})$ определяются собственными (внутренними) параметрами самого агрегата $h(t) \in H$ входными сигналами $x(t) \in X$.

  В начальный момент времени $t_{0}$ состояния z имеют значения, равные $z^{0}$, т. е. $z^{0} = z(t_{0})$, задаваемые законом распределения процесса z(t) в момент времени $t_{0}$, а именно $L[z(t_{0})]$. Предположим, что процесс функционирования агрегата в случае воздействия входного сигнала $x_{n}$ описывается случайным оператором V. Тогда в момент поступления в агрегат $t_{n} \in T$ входного сигнала $x_{n}$ можно определить состояние

  \begin{center}
    $z(t_{n} + 0) = V[t_{n}, z(t_{n}), x_{n}]$
  \end{center}

  Обозначим полуинтервал времени $t_{1} < t <= t_{2}$ как $(t_{1}, t_{2}]$ , а полуинтервал $t_{1} <= t < t_{2}$— как $[t_{1}, t_{2})$. Если интервал времени $(t_{n}, t_{n + 1})$ не содержит ни одного момента поступления сигналов, то для $t \in (t_{n}, t_{n + 1})$ состояние агрегата определяется случайным оператором U в соответствии с соотношением

  \begin{center} $z(t) = U[t, t_{n}, z(t_{n} + 0)]$
  \end{center}

  Совокупность случайных операторов V и U рассматривается как оператор переходов агрегата в новые состояния. При этом процесс функционирования агрегата состоит из скачков состояний $\delta z$ в моменты поступления входных сигналов х (оператор V) и изменений состояний между этими моментами $t_{n}$ и $t_{n} + 1$ (оператор U). На оператор U не накладывается никаких ограничений, поэтому допустимы скачки состояний $\delta z$ в моменты времени, не являющиеся моментами поступления входных сигналов х. В дальнейшем моменты скачков $\delta z$ будем называть особыми моментами времени $t_{\delta}$, а состояния $z(t_{\delta})$ -- особыми состояниями А-схемы. Для описания скачков состояний $\delta z$ в особые моменты времени $t_{\delta}$ будем использовать случайный оператор W, представляющий собой частный случай оператора U, т. е.
  \begin{center}
    $z(t_{\delta} + 0) = W[t_{\delta}, z(t_{\delta})]$
  \end{center}

  В множестве состояний Z выделяется такое подмножество $Z^{(Y)}$, что если $z(t_{\delta})$ достигает $Z^{(Y)}$, то это состояние является моментом выдачи выходного сигнала, определяемого оператором выходов

  \begin{center}
    $y = G[t_{\delta}, z(t_{\delta})]$
  \end{center}

  Таким образом, под агрегатом будем понимать любой объект, определяемый упорядоченной совокупностью рассмотренных множеств T, X, Y, Z, $Z^{(Y)}$, H и случайных операторов V, U, W, G.

  Последовательность входных сигналов, расположенных в порядке их поступления в А-схему, будем называть входным сообщением или х-сообщением.Последовательность выходных сигналов, упорядоченную относительно времени выдачи, назовем выходным сообщением или у-сообщением.


\subsection{Имитационное моделирование.}

    Имитационное моделирование есть процесс конструирования модели реальной системы и постановки эксперимента на этой модели с целью либо понять поведение, либо оценить( в рамках ограничений, накладываемых некоторым критерием или совокупностью критериев) различные стратегии, обеспечивающие функционирование данной системы. Таким образом процесс имитационного моделирования мы понимаем как процесс, включающий и конструирование модели и, и аналитическое применение модели для изучения некоторой проблемы. Под моделью реальной системы понимается представление группы объектов или идей в некоторой форме, отличной от их реального воплощения; отсюда термин "реальный" используется в смысле "существующий или способный принять одну из форм существования".

    Имитационное моделирование является экспериментальной и прикладной методологией, имеющей целью:

\begin{itemize}
    \item описать поведение системы.
    \item построить теории и гипотезы, которые могут объяснить наблюдаемой поведение;
    \item использовать эти теории для предсказания будущего поведения системы, т.е. тех воздействий, которые могут быть вызваны изменениями в системе или изменениями способов ее функционирования.
\end{itemize}

  Идея представления некоторого объекта или понятия при помощи модели носит столь общий характер, что дать полную классификацию всех функций модели затруднительно. Можно выделить по крайней мере пять ставших привычными случаев применения моделей в качестве:

\begin{itemize}
    \item средства осмысления действительности,
    \item средства общения,
    \item средства обучения и тренажа,
    \item инструмента прогнозирования,
    \item средства постановки экспериментов.
\end{itemize}

    Все имитационные модели представляют собой модели типа так называемого черного ящика. Это означает, что они обеспечивают выдачу выходного сигнала системы, если на ее взаимодействующие подсистемы воздействует входной сигнал. Поэтому для получения необходимой информации или результатов необходимо осуществлять "прогон" имитационных моделей, а не "решать" их. Имитационные модели не способны формировать свое собственное решение в том виде, в котором это имеет место в аналитических моделях, а могут лишь служить в качестве средства для анализа поведения систем в условиях, которые определяются экспериментатором. Следовательно имитационное моделирование -- не теория, а методология решения проблем.
    Имитационное моделирование есть экспериментирование с моделью реальной системы. Необходимость решения задачи путем экспериментирования становится очевидной, когда возникает потребность получить о системе специфическую информацию, которую нельзя найти в известных источниках. Непосредственное экспериментирование с на реальной системе устраняет множество затруднений, если необходимо обеспечить соответствие между моделью и реальными условиями; однако недостатки такого экспериментирования иногда весьма значительны, поскольку:

\begin{enumerate}
    \item Оно может нарушить установленный порядок работы фирмы.
    \item Если составной частью системы являются люди, то на результаты экспериментов может повлиять так называемый хауторнский эффект, проявляющийся в том, что люди, чувствуя, что за ними наблюдают, могут изменить свое поведение.
    \item Может оказаться сложным поддержание одних и тех же рабочих условий при каждом повторении эксперимента, или в течении всего времени проведении серии экспериментов.
    \item Для получения одной и той же величины выборки(и, следовательно, статистической значимости результатов эксперимента) могут потребоваться чрезмерные затраты времени и средств.
    \item При экспериментировании с реальными системами может оказаться невозможным исследование множества альтернативных вариантов.
\end{enumerate}

    При этом необходимо рассмотреть целесообразность применения имитационного моделирования при наличии любого из следующих условий:

\begin{enumerate}
    \item Не существует законченной математической постановки данной задачи, либо еще не разработаны аналитические методы решения сформулированной математической модели.
    \item Аналитические методы имеются, но математические процедуры столь сложны, что имитационное моделирование дает более простой способ решения задачи.
    \item Аналитические решения существуют, но их реализация невозможна в следствие не достаточной математической подготовки имеющегося персонала.
    \item Кроме оценки определенных параметров, желательно осуществить на имитационной модели наблюдение за ходом процесса в течение определенного времени.
    \item Имитационное моделирование может оказаться единственной возможностью вследствие трудностей трудностей постановки экспериментов и наблюдения явлений в реальных условиях.
    \item Для долговременного действия систем или процессов может понадобиться сжатие временной шкалы.
\end{enumerate}

\subsubsection{Виды имитационного моделирования.}

    По способу описания процессов моделируемой системы можно выделить три вида имитационного моделирования:

\begin{itemize}
    \item Агентное моделирование -- используется для исследования децентрализованных систем, динамика функционирования которых определяется не глобальными правилами и законами (как в других парадигмах моделирования), а наоборот, когда эти глобальные правила являются результатом индивидуальной активности членов группы. Цель агентных моделей -- получить представление об этих глобальных правилах, общем поведении системы, исходя из предположений об индивидуальном, частом поведении ее ее отдельных активных объектов и взаимодействии этих объектов в системе. Агент -- некая сущность, обадающая активностью, автономным поведением, может принимать решения в соответствии с некоторым набором правил, взаимодействовать с окружением, а так же самостоятельно изменяться.
    \item Дискретно-событийное моделирование -- подход к моделированию, предлагающий абстрагироваться от непрерывной природы событий и рассматривать только основные события моделируемой системы, такие как "ожидание", "обработка заказа", "движение с грузом", "разгрузка" и другие. Дискретно-событийное моделирование наиболее развито и имеет огромную сферу приложений -- от логистики и систем массового обслуживания до транспортных и производственных систем. Этот вид моделирования наиболее подходит для моделирования производственных процессов.
    \item Системная-динамика - парадигма моделирования, где для исследуемой системы строятся графические диаграммы причинных связей и глобальных влияних одних параметров на другие во времени, а затем созданная на основе этих диаграмм модель имитируется на компьютере. По сути, такой вид моделирования позволяет более всех других парадигм понять суть происходящего выявления причинно-следственных связей между объектами и явлениями.
\end{itemize}

\subsection{Языки имитационного моделирования.}

    Для моделирования систем используются как универсальные и процедурно-ориентированные языки общего назначения(ЯОН), так и специализированные языки имитационного моделирования(ЯИМ). При этом ЯОН предоставляют программисту-разработчику модели $M_{M}$ больше возможностей в смысле гибкости разработки, отладки и использования модели. Но гибкость приобретается ценой больших усилий, затрачиваемых на программирование модели, так как организация выполнения операций, отсчет системного времени и контроль хода вычислений существенно усложняются.

    Имеющиеся ЯИМ можно разбить на три основные группы, соответствующие трем типам математических схем: непрерывные, дискретные и комбинированные. Языки каждой группы предназначены для соответствующего представления системы S при создании ее машинной модели $M_{M}$.

    В основе рассматриваемой классификации в некоторых ЯИМ лежит принцип формирования системного времени. Так как «системные часы» предназначены не только для продвижения системного времени в модели Мм, но также для синхронизации различных событий и операций в модели системы S, то при отнесении того или иного конкретного языка моделирования к определенному типу нельзя не считаться с типом механизма «системных часов».

    Непрерывное представление системы S сводится к составлению уравнений, с помощью которых устанавливается связь между эндогенными и экзогенными переменными модели. Примером такого непрерывного подхода является использование дифференциальных уравнений. Причем в дальнейшем дифференциальные уравнения могут быть применены для непосредственного получения характеристик системы, это, например, реализовано в языке MIMIC. А в том случае, когда экзогенные переменные модели принимают дискретные значения, уравнения являются разностными. Такой подход реализован, например, в языке DYNAMO.

    Представление системы S в виде типовой схемы, в которой участвуют как непрерывные, так и дискретные величины, называется комбинированным. Примером языка, реализующего комбинированный подход, является GASP, построенный на базе языка FORTRAN. Язык GASP включает в себя набор программ, с помощью которых моделируемая система S представляется в следующем виде. Состояние модели системы М (S) описывается набором переменных, некоторые из которых меняются во времени непрерывно. Законы изменения непрерывных компонент заложены в структуру, объединяющую дифференциальные уравнения и условия относительно переменных. Предполагается, что в системе могут наступать события двух типов: 1) события, зависящие от состояния $z_{j}$, 2) события, зависящие от времени $t_{i}$. События первого типа наступают в результате выполнения условий, относящихся к законам изменения непрерывных переменных. Для событий второго типа процесс моделирования состоит в продвижении системного времени от момента наступления события до следующего аналогичного момента. События приводят к изменениям состояния модели системы и законов изменения непрерывных компонент. При использовании языка GASP на пользователя возлагается работа по составлению на языке FORTRAN подпрограмм, в которых он описывает условия наступления событий, зависящих от процесса функционирования системы S, законы изменения непрерывных переменных, а также правила перехода из одного состояния в другое.

    В рамках дискретного подхода можно выделить несколько принципиально различных групп ЯИМ. Первая группа ЯИМ подразумевает наличие списка событий, отличающих моменты начала выполнения операций. Продвижение времени осуществляется по событиям, в моменты наступления которых производятся необходимые операции, вклю­
чая операции пополнения списка событий. Примером языка событий является язык SIMSCRIPT. Разработчики языка SIMSCRIPT исходили из того, что каждая модель $M_{M}$ состоит из элементов, с которыми происходят события, представляющие собой последовательность предложений, изменяющих состояния моделируемой системы в различные моменты времени.

    При использовании ЯИМ второй группы после пересчета системного времени, в отличие от схемы языка событий, просмотр действий с целью проверки выполнения условий начала или окончания какого-либо действия производится непрерывно. Просмотр действий определяет очередность появления событий. Языки данного типа имеют в своей основе поисковый алгоритм, и динамика системы S описывается в терминах действий. Примером языка действий (работ) является ЯИМ FORSIM, представляющий собой пакет прикладных программ, который позволяет оперировать только фиксированными массивами данных, описывающих объекты моделируемой системы. С его помощью нельзя имитировать системы переменного состава. При этом размеры массивов  устанавливаются либо во время компиляции программы, либо в самом начале ее работы. Язык FORSIM удобен для описания систем с большим числом разнообразных ресурсов, так как он позволяет записывать условия их доступности в компактной форме. Конкретный способ формализации модели на языке действий в достаточной степени произволен и остается на усмотрение программиста, что требует его достаточно высокой квалификации. Полное описание динамики модели $M_{M}$ можно получить с помощью разных наборов подпрограмм.

    Третья группа ЯИМ описывает системы, поведение которых определяется процессами. В данном случае под процессом понимается последовательность событий, связь между которыми устанавливается с помощью набора специальных отношений. Динамика заложена в независимо управляемых программах, которые в совокупности составляют программу процесса. Пример языка процессов — язык SIMULA, в котором осуществляется блочное представление моделируемой системы S c использованием понятия процесса для формализации элементов, на которые разбивается моделируемая система. Процесс задается набором признаков, характеризующих его структуру, и программой функционирования. Функционирование каждого процесса разбивается на этапы, протекающие в системном времени.

    Главная роль в языке SIMULA отводится понятию параллельного оперирования с процессами в системном времени, а также универсальной обработке списков с процессами в роли компонент. Специальные языковые средства предусмотрены для манипуляций с упорядоченными множествами процессов.

    В отдельную группу могут быть выделены ЯИМ типа GPSS, хотя принципиально их можно отнести к группе языков процессов. Язык GPSS представляет собой интерпретирующую языковую систему, применяющуюся для описания  пространственного движения объектов. Такие динамические объекты в языке GPSS называются транзактами и представляют собой элементы потока. В процессе имитации транзакты «создаются» и «уничтожаются». Функцию каждого из них можно представить как движение через модель $M_{M}$ с поочередным воздействием на ее блоки. Функциональный аппарат языка образуют блоки, описывающие логику модели, сообщая транзактам, куда двигаться и что делать дальше. Данные для ЭВМ подготавливаются в виде пакета управляющих и определяющих карт, который составляется по схеме модели, набранной из стандартных символов. Созданная GPSS-программа, работая в режиме интерпретации, генерирует и передает транзакты из блока в блок в соответствии с правилами, устанавливаемыми блоками. Каждый переход транзакта приписывается к  определенному моменту системного времени.


\subsection{Описание разрабатываемой системы.}

    Проанализировав приведенное описание языков имитационного моделирования можно сделать вывод о том, что они обладают рядом недостатков, которые не позволяют использовать их в качестве решения поставленной задачи. Использование ЯИМ при проектировании сети требует от специалиста по безопасности определенных знаний в области моделирования, кроме того, разработанные специально для решения определенного класса задач, они не предоставляют интерфейса, позволяющего в короткие сроки решить поставленную задачу. При использовании ЯИМ, специалисту по безопасности необходимо построить адекватную модель предметной области, провести эксперименты и, по результатам эксперимента, вынести решение о пригодности использования выбранного решения, или изменить модель и провести новую серию испытаний.
    Разрабатываемая система должна обладать возможностью создания адекватной модели предметной области с минимальными временными затратами. Под адекватной моделью понимается модель, достаточно точно отражающая характер процессов, протекающих в предметной области. Таким образом можно выделить два главных требования к разрабатываемой системе:

\begin{itemize}
    \item удобство построения модели исследуемого объекта;
    \item адекватность построенной модели;
\end{itemize}

    Удобство построения модели заключается в том, чтобы пользователь мог использовать готовые модели объектов описывая только порядок их взаимодействия. С другой стороны, может возникнуть необходимость реализации нового поведения объекта или описание нового класса объектов. С этой точки зрения, система должна обладать достаточной гибкостью и расширяемостью, чтобы без значительных изменений описать необходимый объект. Т.е. система должна обладать такими возможностями как:

\begin{itemize}
    \item возможность использования объектов высокого уровня для построения модели исследуемой системы;
    \item возможность гибкой настройки поведения используемых объектов;
    \item возможность описания новых объектов или введение новых моделей поведения;
    \item возможность мониторинга состояния объектов с целью выявления возникающих проблем;
\end{itemize}

    Таким образом разрабатываемую систему можно разбить на несколько подсистем.

    Подсистема управления процессом моделирования. Эта подсистема отвечает за взаимодействие между моделью и управляющей средой. Данная система необходима для хранения конфигураций моделей отдельных объектов, создание модели предметной области, настройку взаимодействия между объектами в модели. Так же предоставляет возможность пользователю вносить изменения в состав и настройки модели. управляет процессом моделирования, реализует механизмы, которые не имеют отношения к модели, но используются ею в процессе работы.

    Модель. Модель предметной области разбивается на несколько уровней. Верхний из которых отвечает за передачу управляющего воздействия отдельным компонентам, а так же служит для создания и настройки компонентов более низкого уровня. Модели более низкого уровня являются моделями отдельных объектов, участвующих во взаимодействии или отдельными частями таких компонентов. Подобный подход позволит обеспечить достаточную для построения адекватной модели точность описания процессов.

    Подсистема мониторинга. Данная подсистема позволяет контролировать состояние объектов. Для реализации данной подсистемы необходимо выделить параметры, являющиеся важными для функционирования исследуемой системы, и определить процесс обработки данных(Рисунок ??).


  %Описание способов моделирования процессов

        %Сравнительный анализ описанных способов.
        %Системы моделирования процессов. Сложность использования для задачи.
        %Почему не использую языки моделирования
        %Почему не использую готовые решения.
      %Безопасность
        %Возможные способы нарушения безопасности с использованием локальной сети.
        %Подходы к анализу безопасности в локальных сетях.
        %Описание различных моделей для оценки безопасности.
        %Сравнительный анализ описанных моделей.
      %Итог. Выбор способа моделирования. Описание требований.
